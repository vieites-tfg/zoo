\pagestyle{plain}
\chapter*{Resumen}

Ante la creciente complejidad de los flujos DevOps, resulta conveniente identificar herramientas de \textit{Continuous Integration/Continuous Delivery} (CI/CD) que unifiquen los \textit{pipelines} y reduzcan los costes de mantenimiento. Este trabajo investiga si Dagger, un motor programable basado en contenedores, puede simplificar y acelerar la integración y entrega continuas frente a los métodos convencionales predominantes.

Se construyó un \textit{monorepo} con una aplicación de prueba (\textit{frontend}: Vue + Typescript; \textit{backend}: Typescript; MongoDB) y se diseñaron e implementaron dos módulos de Dagger para los ciclos de CI y CD, empleando el SDK que proporciona Dagger para el lenguaje de programación Go. Además, se desplegó la aplicación en tres \textit{clusters} de KinD, gestionados por ArgoCD bajo filosofía GitOps, con la infraestructura definida mediante Helm y Kubernetes.

La evaluación cuantitativa incluyó una prueba en la que se ejecutó la función \texttt{endtoend} del módulo de CI. Esta mostró una reducción del tiempo de ejecución del 40\% cuando los parámetros de entrada cambiaban entre ejecuciones, y del 94\% cuando los parámetros permanecían sin cambios. Esto es gracias al sistema de gestión de caché nativo de Dagger. El análisis cualitativo destacó ventajas en cuanto al mantenimiento, la portabilidad (ejecución sobro \textit{runtime} OCI), la reproducibilidad y la escalabilidad, frente a una curva de aprendizaje moderada.

En conclusión, Dagger se posiciona como una alternativa sólida para estandarizar y modernizar los ciclos de vida del \textit{software}, facilitando la integración continua y reduciendo la dependencia de \textit{scripts} heterogénos. Se sugiere como mejora futura la validación del funcionamiento de los módulos en \textit{clusters} gestionados en la nube.
