\markboth{BIBLIOGRAFÍA}{BIBLIOGRAFÍA}
\addcontentsline{toc}{chapter}{Bibliografía}

\begin{thebibliography}{99}
\bibitem{dagger} Dagger.io. ``Dagger Documentation | Dagger.'' Dagger.io, 2022, {\it https://docs.dagger.io}
\bibitem{blog} Dagger.io. “Dagger | Blog.” Dagger.io, 2025, {\it https://dagger.io/blog/}. Accedido el 15 de junio de 2025

\bibitem{ci} Fowler, Martin. ``Continuous Integration.'' Martinfowler.com, 18 Jan. 2024, {\it https://martinfowler.com/articles/continuousIntegration.html}.
\bibitem{img:ci} PagerDuty, Inc. ``What Is Continuous Integration?'' PagerDuty, 20 Nov. 2020, {\it https://www.pagerduty.com/resources/devops/learn/what-is-continuous-integration}.
\bibitem{img:cd} Amazon Web Services, Inc. ``¿Qué Es La Entrega Continua? – Amazon Web Services.'' Amazon Web Services, Inc., 2024, {\it https://aws.amazon.com/es/devops/continuous-delivery}.

\bibitem{cd} Fowler, Martin. ``Bliki: ContinuousDelivery.'' Martinfowler.com, 2013, {\it https://martinfowler.com/bliki/ContinuousDelivery.html}.

\bibitem{monorepo} Nx. ``Monorepo Explained.'' Monorepo.tools, 2025, {\it https://monorepo.tools}
\bibitem{kubernetes} Los autores de Kubernetes. ``Orquestación de Contenedores Para Producción.'' Kubernetes, 2025, {\it https://kubernetes.io/es}.
\bibitem{helm} Helm. ``Helm.'' Helm.sh, 2019, {\it https://helm.sh}.
\bibitem{devops} Atlassian. ``¿Qué Es DevOps?'' Atlassian, {\it https://www.atlassian.com/es/devops}.

\bibitem{push} Nasser, Mohammed. ``Push vs. Pull-Based Deployments.'' DEV Community, 25 Nov. 2024, {\it https://dev.to/mohamednasser018/push-vs-pull-based-deployments-4m78}. Accedido el 14 de junio de 2025.
\bibitem{git} Git. ``Git.'' Git-Scm.com, 2024, {\it https://git-scm.com}.
\bibitem{make} Wikipedia Contributors. ``Make (Software).'' Wikipedia, 10 July 2021, en {\it https://en.wikipedia.org/wiki/Make\_(software)}.
\bibitem{just} casey. ``GitHub - Casey/Just: Just a Command Runner.'' GitHub, 2025, {\it https://github.com/casey/just}. Accedido el 14 de junio de 2025.
\bibitem{shebang} Wikipedia Contributors. ``Shebang (Unix).'' Wikipedia, 13 Aug. 2021, {\it https://en.wikipedia.org/wiki/Shebang\_(Unix)}.

\bibitem{devops} Wikipedia Contributors. ``DevOps.'' Wikipedia, Wikimedia, 1 Dec. 2019, {\it https://en.wikipedia.org/wiki/DevOps}.
\bibitem{docker} Wikipedia Contributors. ``Docker (Software).'' Wikipedia, 16 Nov. 2019, {\it https://en.wikipedia.org/wiki/Docker\_(software)}.
\bibitem{docker-compose} ``Overview of Docker Compose.'' Docker Documentation, 10 Feb. 2020, {\it https://docs.docker.com/compose}.
\bibitem{img:k8s} Kong. ``What Is Kubernetes? Examples and Use Cases.'' Kong Inc., 2024, {\it https://konghq.com/blog/learning-center/what-is-kubernetes}.
\bibitem{cluster} The Kubernetes Authors. ``Cluster Architecture.'' Kubernetes, 2025, {\it https://kubernetes.io/docs/concepts/architecture}.
\bibitem{kind} The Kubernetes Authors. ``Kind.'' Kind.sigs.k8s.io, 2025, {\it https://kind.sigs.k8s.io}.
\bibitem{kubectl} The Kubernetes Authors. ``Command Line Tool (Kubectl).'' Kubernetes, 2025, {\it https://kubernetes.io/docs/reference/kubectl}.
\bibitem{gitops} GitLab. ``¿Qué Es GitOps?'' Gitlab.com, GitLab, 9 Feb. 2022, {\it https://about.gitlab.com/es/topics/gitops}. Accedido el 20 de junio de 2025.
\bibitem{argocd} Argo Porject Authors. ``Argo CD.'' Github.io, 2025, {\it https://argoproj.github.io/cd}. Accedido el 20 de junio de 2025.
\bibitem{img:helm} Flexera. ``Kubernetes Helm: K8s Application Deployment Made Simple.'' Spot.io, 12 Sept. 2024, {\it https://spot.io/resources/kubernetes-architecture/kubernetes-helm-k8s-application-deployment-made-simple}. Accedido el 20 de junio de 2025.

\bibitem{img:dagger} Dagger. ``Introducing Dagger: A New Way to Create CI/CD Pipelines | Dagger.'' Dagger.io, 2022, {\it https://dagger.io/blog/public-launch-announcement}. Accedido el 21 de junio de 2025.
\bibitem{oci} Wikipedia Contributors. ``Open Container Initiative.'' Wikipedia, Wikimedia Foundation, 12 Nov. 2024, {\it https://en.wikipedia.org/wiki/Open\_Container\_Initiative}.
\bibitem{cue} CUE. ``The CUE Language Specification.'' CUE, 16 Apr. 2025, {\it https://cuelang.org/docs/reference/spec}. Accedido el 21 de junio de 2025.
\bibitem{sdk} Wikipedia Contributors. ``Conjunto de Herramientas de Desarrollo de Software.'' Wikipedia, 15 Aug. 2006, {\it https://es.wikipedia.org/wiki/Kit\_de\_desarrollo\_de\_software}. Accedido el 21 de junio de 2025.
\bibitem{github-actions} GitHub. ``Features $\cdot$ GitHub Actions.'' GitHub, 2025, {\it https://github.com/features/actions}.
\bibitem{go} ``The Go Programming Language.'' Go.dev, 2025, {\it https://go.dev}. Accedido el 21 junio de 2025.
\bibitem{graphql} The GraphQL Foundation. ``GraphQL: A Query Language for APIs.'' Graphql.org, 2012, {\it https://graphql.org}.
\bibitem{daggergo} dagger. ``Examples/Go/Yarn-Build/Ci.go at Main · Dagger/Examples.'' GitHub, 2023, {\it https://github.com/dagger/examples/blob/main/go/yarn-build/ci.go}. Accedido el 21 de junio de 2025.

\bibitem{img:graphql} Dagger. ``Introducing the Dagger GraphQL API | Dagger.'' Dagger.io, 2022, {\it https://dagger.io/blog/graphql}. Accedido el 21 de junio de 2025.
\bibitem{daggerverse} Dagger. ``Daggerverse.'' Daggerverse.dev, 2025, {\it https://daggerverse.dev}. Accedido el 21 de junio de 2025.
\bibitem{cli} Dagger. ``ContentKeeper Content Filtering.'' Dagger.io, 2025, {\it https://docs.dagger.io/reference/cli}. Accedido el 21 de junio de 2025
\bibitem{ts} Microsoft. ``TypeScript - JavaScript That Scales.'' Typescriptlang.org, 2025, {\it https://www.typescriptlang.org}.
\bibitem{lerna} Nx. ``Lerna · a Tool for Managing JavaScript Projects with Multiple Packages.'' Lerna.js.org, 2025, {\it https://lerna.js.org}.
\bibitem{mongodb} MongoDB. ``MongoDB.'' MongoDB, 2024, {\it https://www.mongodb.com}.
\bibitem{vue} You, Evan. ``Vue.js.'' Vuejs.org, 2014, {\it https://vuejs.org}.
\bibitem{linear} LinearB, Inc. ``Cycle Time Breakdown: Tactics for Reducing PR Review Time | LinearB Blog.'' Linearb.io, 4 Aug. 2021, {\it https://linearb.io/blog/reducing-pr-review-time}. Accedido el 10 de julio de 2025.

\end{thebibliography}

