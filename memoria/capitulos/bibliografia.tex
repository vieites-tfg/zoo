\markboth{BIBLIOGRAFÍA}{BIBLIOGRAFÍA}
\addcontentsline{toc}{chapter}{Bibliografía}


\begin{thebibliography}{99}
% EXEMPLO DE DOCUMENTO DESCARGADO DA WEB
\bibitem{dagger} Dagger.io. ``Dagger Documentation | Dagger.'' Dagger.io, 2022, {\it https://docs.dagger.io/}
\bibitem{blog} Dagger.io. “Dagger | Blog.” Dagger.io, 2025, {\it https://dagger.io/blog/}. Accessed 15 June 2025

\bibitem{ci} Fowler, Martin. ``Continuous Integration.'' Martinfowler.com, 18 Jan. 2024, {\it https://martinfowler.com/articles/continuousIntegration.html}
\bibitem{img.ci} PagerDuty, Inc. ``What Is Continuous Integration?'' PagerDuty, 20 Nov. 2020, {\it https://www.pagerduty.com/resources/devops/learn/what-is-continuous-integration/}

\bibitem{cd} Fowler, Martin. ``Software Delivery Guide.'' Martinfowler.com, 21 Aug. 2019, {\it https://martinfowler.com/delivery.html}. Accessed 14 June 2025.

\bibitem{monorepo} Nx. ``Monorepo Explained.'' Monorepo.tools, 2025, {\it https://monorepo.tools}
\bibitem{kubernetes} Los autores de Kubernetes. ``Orquestación de Contenedores Para Producción.'' Kubernetes, 2025, {\it https://kubernetes.io/es}
\bibitem{helm} Helm. ``Helm.'' Helm.sh, 2019, {\it https://helm.sh}
\bibitem{devops} Atlassian. ``¿Qué Es DevOps?'' Atlassian, {\it https://www.atlassian.com/es/devops}

\bibitem{push} Nasser, Mohammed. ``Push vs. Pull-Based Deployments.'' DEV Community, 25 Nov. 2024, {\it https://dev.to/mohamednasser018/push-vs-pull-based-deployments-4m78}. Accessed 14 June 2025.
\bibitem{git} Git. ``Git.'' Git-Scm.com, 2024, {\it https://git-scm.com}
\bibitem{make} ``Make (Software).'' Wikipedia, 10 July 2021, en {\it https://en.wikipedia.org/wiki/Make\_(software)}
\bibitem{just} casey. ``GitHub - Casey/Just: Just a Command Runner.'' GitHub, 2025, {\it https://github.com/casey/just}. Accessed 14 June 2025.
\bibitem{shebang} Wikipedia. ``Shebang (Unix).'' Wikipedia, 13 Aug. 2021, {\it https://en.wikipedia.org/wiki/Shebang\_(Unix)}


% EXEMPLO DE PÁXINA DA WIKIPEDIA
% \bibitem{cdma} Acceso múltiple por división de código. Artigo da wikipedia ({\it http://es.wikipedia.org}). Consultado o 2 de xaneiro do 2010.

% EXEMEPLO DE LIBRO
% \bibitem{gonzalez} R.C. Gonzalez e R.E. Woods, {\it Digital image processing}, 3ª edición, Prentice Hall, New York, 2007.

% EXEMPLO DE ARTIGO DE REVISTA
% \bibitem{patricia} P. González, J.C. Cartex e T.F. Pelas, ``Parallel computation of wavelet transforms using the lifting scheme'', {\it Journal of Supercomputing}, vol. 18, no. 4, pp. 141-152, junio 2001.
\end{thebibliography}

