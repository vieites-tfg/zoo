\markboth{BIBLIOGRAFÍA}{BIBLIOGRAFÍA}
\addcontentsline{toc}{chapter}{Bibliografía}


\begin{thebibliography}{99}
% EXEMPLO DE DOCUMENTO DESCARGADO DA WEB
\bibitem{dagger} Dagger.io. (2024). Official Dagger Documentation. Recuperade de {\it https://docs.dagger.io/}.

\bibitem{ci} Fowler, M. (2024). Continuous Integration. Recuperado de {\it https://martinfowler.com/articles/continuousIntegration.html}.
\bibitem{img.ci} PagerDuty, Inc. (2025). What is Continuous Integration? Recuperado de {\it https://www.pagerduty.com/resources/devops/learn/what-is-continuous-integration/}.

\bibitem{cd} Fowler, M. (2019). Software Delivery Guide. Recuperado de {\it https://martinfowler.com/delivery.html}.

\bibitem{monorepo} Nx (2025). Everyting you need to know about monorepos, and the tools to build them. Recuperado de {\it https://monorepo.tools/}
\bibitem{kubernetes} Los autores de Kubernetes (2025). Orquestaciónn de contenedores para producción. Recuperado de {\it https://kubernetes.io/es/}
\bibitem{helm} Helm authors (2025). The package manager por Kubernetes. Recuperado de {\it https://helm.sh/}
\bibitem{devops} Atlassian (2025). ¿Qué es DevOps?. Recuperado de {\it https://www.atlassian.com/es/devops}

\bibitem{push} Mohammed Nasser (2024). Push vs. Pull-Based Depoyments. Recuperado de {\it https://dev.to/mohamednasser018/push-vs-pull-based-deployments-4m78}


% EXEMPLO DE PÁXINA DA WIKIPEDIA
% \bibitem{cdma} Acceso múltiple por división de código. Artigo da wikipedia ({\it http://es.wikipedia.org}). Consultado o 2 de xaneiro do 2010.

% EXEMEPLO DE LIBRO
% \bibitem{gonzalez} R.C. Gonzalez e R.E. Woods, {\it Digital image processing}, 3ª edición, Prentice Hall, New York, 2007.

% EXEMPLO DE ARTIGO DE REVISTA
% \bibitem{patricia} P. González, J.C. Cartex e T.F. Pelas, ``Parallel computation of wavelet transforms using the lifting scheme'', {\it Journal of Supercomputing}, vol. 18, no. 4, pp. 141-152, junio 2001.
\end{thebibliography}

