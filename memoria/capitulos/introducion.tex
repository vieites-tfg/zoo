\chapter{Introducción}

% Obxectivos Xerais, Relación da Documentación que conforma a Memoria, Descrición do Sistema, Información Adicional de Interese (métodos, técnicas ou arquitecturas utilizadas, xustificación da súa elección, etc.).

\section{Objetivos generales}

% qué problema se resuelve y cuál es el propósito de este trabajo

\subsection*{Objetivos principales}

En este trabajo se pretende demostrar y evaluar la viabilidad, eficiencia y flexibilidad de Dagger\cite{dagger} como motor de CI/CD (\textit{Continuous Integration}/\textit{Continuous Delivery})\cite{ci,cd}, con el fin de estandarizar y modernizar los ciclos de vida del desarrollo de software.

Se usará Dagger con el fin de implementar la lógica necesaria para llevar a cabo los ciclos de CI y CD de una aplicación de prueba. Se evaluarán las ventajas que tiene su uso frente a métodos convencionales, entre las que destacarán su portabilidad, al correr sobre un \textit{runtime} de OCI (\textit{Open Container Initiative}\cite{oci}), como Docker\cite{docker}; y su capacidad programática, ya que se pueden crear \textit{pipelines} implementando funciones en el lunguaje conocido para el desarrollador. En vez de coordinar \textit{scripts} creados a mano en diferentes entornos, el programador es capaz de componer acciones reusables, utilizando un lenguaje de programación y una API (\textit{Application Programming Interface}) a su disposición.

Se van a proporcionar ejemplos de módulos creados con Dagger, los cuales estarán especialmente diseñados para cumplir los ciclos tanto de CI como de CD de la aplicación de prueba. De esta manera se podrá comprobar que este mismo proceso se puede llevar a cabo para cualquier aplicación.

Lo que sigue son las preguntas a las que este trabajo busca responder:

\begin{itemize}
  \item ¿Qué grado de complejidad tiene desarrollar módulos de Dagger para la gestión de un ciclo de CI/CD de una aplicación?
  \item ¿Vale realmente la pena aprender a utilizar esta herramienta?
  \item ¿Es capaz de aumentar la velocidad de desarrollo de una aplicación?
  \item ¿Es fácilmente integrable en cualquier tipo de aplicación?
  \item ¿Qué puntos débiles corrige Dagger frente al uso de otros métodos convencionales?
  \item ¿Qué desafíos, limitaciones o desventajas se encuentran al trabajar con Dagger?
\end{itemize}

\subsection*{Objetivos secundarios}

Para lograr los objetivos principales es necesario llevar a cabo varios pasos:

\begin{itemize}
  \item Creación de un \textit{monorepo}\cite{monorepo} en GitHub.
  \item Diseño y creación de una aplicación de prueba. Esta consistirá en una página web de gestión de un zoo, la cual realizará peticiones a una API REST que estará conectada a una base de datos.
  \item Implementación de un \textit{pipeline} CI/CD con Dagger.
  \item Entorno orquestado por Kubernetes\cite{kubernetes}, configurado a través de una Chart de Helm\cite{helm}.
  \item Análisis comparativo de las ventajas que ofrece Dagger frente a métodos convencionales.
\end{itemize}

\section{Relación de la documentación}

\begin{itemize}
  \item Capítulo 1: Introducción.

    Este capítulo, en el cual se describen la finalidad del proyecto, las tecnologías a utilizar, de manera breve; y la estructura, a grandes rasgos, del trabajo en sí.

  \item Capítulo 2: Estado del arte y fundamentos teóricos.

    En el segundo capítulo se detallan los conceptos más importantes de CI/CD. Además, se estudia la evolución de las herramientas DevOps\cite{devops}, incluyendo Dagger como una de las últimas y más innovadoras herramientas en este sector.
  \item Capítulo 3: Diseño y arquitectura del sistema.

    Aquí se describe la organización de repositorio, así como las tecnologías utilizadas para implementar cada una de las piezas de \textit{software} del trabajo. Por lo tanto, se habla de la aplicación de prueba y de los módulos de Dagger. También se explica cómo se ha organizado la infraestructura de despliegue.
  \item Capítulo 4: Implementación del \textit{pipeline} con Dagger.

    Aquí se detallan los pasos que se han dado para crear los \textit{pipelines} con Dagger, utilizando el SDK para definirlos como código. Este es el núcleo del proyecto.
  \item Capítulo 5: Pruebas y resultados.

    En este capítulo se presentan las pruebas que se han llevado a cabo. Se habla de las dificultades que se han tenido, así como de las ventajas que ofrece Dagger frente a otras tecnologías, aportando comparaciones cuantitativas y cualitativas.
  \item Capítulo 6: Conclusiones y líneas futuras.

    Finalmente, se resumen los hechos que se han obtenido, se valora el resultado final del uso de Dagger y se indica si ha cumplido con las expectativas. Además, se añaden puntos de mejora o extensiones del proyecto.
\end{itemize}
