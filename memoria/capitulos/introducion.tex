\chapter{Introducción}

% Obxectivos Xerais, Relación da Documentación que conforma a Memoria, Descrición do Sistema, Información Adicional de Interese (métodos, técnicas ou arquitecturas utilizadas, xustificación da súa elección, etc.).

\section{Objetivos generales}

% qué problema se resuelve y cuál es el propósito de este trabajo

\subsection*{Objetivos principales}

En este trabajo se pretende demostrar y evaluar la viabilidad, eficiencia y flexibilidad de Dagger\cite{dagger} como motor de CI/CD (\textit{Continuous Integration}/\textit{Continuous Delivery})\cite{ci,cd}, con el fin de estandarizar y modernizar los ciclos de vida del desarrollo de software.

No solo se va a utilizar Dagger como complemento del ciclo de desarrollo de una aplicación, sino que se va a comparar con la no utilización de este. Se evaluarán las ventajas que tiene su uso frente a métodos convencionales, entre las que destacarán su portabilidad, al poder crear \textit{pipelines} de manera programática, implementando funciones que corren dentro de contenedores, permitiendo al programador tener el control total del entorno en el que se ejecuta la aplicación. En vez de intentar unir \textit{scripts} creados a mano en diferentes entornos, el programador es capaz de componer acciones reusbles, utilizando un lenguaje de programación y una API a su disposición.

Se van a proporcionar ejemplos de módulos creados con Dagger, los cuales estarán especialmente diseñados para cumplir los ciclos tanto de CI  como de CD  de una aplicación \textit{dummy}. De esta manera se podrá comprobar que este mismo proceso se puede llevar a cabo para cualquier aplicación, y de una manera sencilla.

¿Cómo de viable es desarrollar módulos de Dagger para la gestión de un ciclo de CI/CD de una aplicación? ¿Vale realmente la pena aprender a utilizar esta herramienta? ¿Es capaz de aumentar la velocidad de desarrollo de una aplicación?

Estas y otras preguntas se resuelven a lo largo del presente trabajo.

\subsection*{Objetivos secundarios}

Para lograr los objetivos principales es necesario llevar a cabo varios pasos:
\begin{itemize}
  \item Creación de un monorepo\cite{monorepo} en GitHub.

    Todo el código principal se almacenará en un mismo repositorio. De esta manera se evitarán complicaciones debido a la gestión de dependencias de cada una de las partes de la aplicación. Permitirá controlar todo el código fuente de una manera más sencilla al tratarse de un proyecto relativamente pequeño.
  \item Diseño y creación de una aplicación \textit{dummy}.

    Es necesario una aplicación sobre la que realizar los ciclos de CI/CD. Esta consistirá en una página web (frontend) de gestión de un zoo, la cual realizará peticiones a una API (backend) que estará conectada a una base de datos.
  \item Implementación de un \textit{pipeline} CI/CD.

    Se creará un módulo de Dagger para cada uno de los procesos (CI y CD). El módulo de CI permitirá desde compilar la aplicación, hasta publicar las imágenes de Docker y los paquetes NPM de cada una de las partes. Por otro lado, el módulo de CD será el encargado de utilizar esas imágenes que se han publicado previamente y desplegar la aplicación en el entorno correspondiente.
  \item Entorno orquestado.

    El \textit{pipeline} de CD termina con el despliegue de la aplicación sobre Kubernetes\cite{kubernetes}, utilizando Helm\cite{helm}. Esto permite levantar la aplicación en el entorno que le corresponda, según el estado en el que se encuentra cada versión.

  \item Análisis comparativo

    Finalmente, se realiza un análisis de las ventajas que ofrece Dagger frente a métodos convencionales.
\end{itemize}

\section{Relación de la documentación}

\begin{itemize}
  \item Capítulo 1: Introducción.

    Este capítulo, en el cual se describen la finalidad del proyecto, las tecnologías a utilizar, de manera breve; y la estructura, a grandes rasgos, del trabajo en sí.

  \item Capítulo 2: Estado del arte y fundamentos teóricos.

    En el segundo capítulo se detallan los conceptos más importantes de CI/CD. Además, se estudia la evolución de las herramientas DevOps\cite{devops}, incluyendo Dagger como una de las últimas y más innovadoras herramientas en este sector, y se compara con las otras tecnologías.
  \item Capítulo 3: Diseño y arquitectura del sistema.

    Aquí se describen las tecnologías utilizadas para implementar la aplicación \textit{dummy}. También se explica cómo se ha organizado la infraestructura de despliegue.
  \item Capítulo 4: Implementación del \textit{pipeline} con Dagger.

    Aquí se indican los pasos que se han dado para crear los \textit{pipelines} con Dagger, utilizando el SDK para definirlos como código. Este es el núcleo del proyecto.
  \item Capítulo 5: Pruebas y resultados.

    En este capítulo se presentan las pruebas que se han llevado a cabo. Se habla de las dificultades que se han tenido, así como de las ventajas que ofrece Dagger frente a otras tecnologías, aportando comparaciones cuantitativas y cualitativas.
  \item Capítulo 6: Conclusiones y líneas futuras.

    Finalmente, se resumen los hechos que se han obtenido, se valora el resultado final del uso de Dagger y se indica si ha cumplido con las expectativas. Además, se añaden puntos de mejora o extensiones del proyecto.
\end{itemize}
