 
% Para una visualizacion correcta, generar el PDF
% Ni el DVI ni el PS se visualizan bien

% Elegir el estilo que se desee, hay cientos en la red

\documentclass{beamer}
% \usepackage{beamerthemeshadow}
\usetheme{Dresden}
% \usepackage[galician]{babel}
\usepackage[spanish]{babel}
\usepackage[utf8]{inputenc}
%\usepackage[latin1]{inputenc}

\begin{document}
\title{Ciclo completo de CI/CD con Dagger y Kubernetes}
\subtitle{Grado en Ingeniería Informática \\
    Universidad de Santiago de Compostela}
\author{Autor: Daniel Vieites Torres}
\institute{Tutor: Juan Carlos Pichel Campos \\ Co-tutor: Francisco Maseda Muiño}
\date{\today}

\begin{frame}
    \titlepage
\end{frame}

\begin{frame}
    \frametitle{Tabla de contenidos}\tableofcontents
\end{frame}

\section{Objetivos}
\begin{frame}
    \frametitle{¿Cuál es el problema?}
    \begin{columns}
        \begin{column}{0.45\textwidth}
            \begin{itemize}
                \item<1-> Aplicaciones con un gran volumen de tecnologías.
                      % - frontend, proxy, backend, BDs
                \item<2-> Gran complejidad.
                      % sintaxis
                      % tipos de valores
                      % número de yaml en gitlab
                \item<3-> Coste de mantenimiento elevado.
                      % scripts heterogeneos, copia-pega
                      % poca capacidad de abstracción
                      % versionado de herramientas
                \item<4-> Baja portabilidad.
                      % ejecución en remoto
                      % sin pipeline local, sin docker
                      % dependencia de la plataforma
                      % push and pray
            \end{itemize}
        \end{column}
        \begin{column}{0.55\textwidth}
            \begin{figure}
                \includegraphics<1>[height=4.9cm]{figuras/Gitlab}
                \only<1>{\caption{Arquitectura de GitLab\cite{gitlab_architecture}.}}
                \includegraphics<2>[scale=0.4]{figuras/Gitlab_yamls}
                \only<2>{\caption{Archivos YAML en GitLab\cite{gitlab_repo}.}}
                \includegraphics<3>[scale=0.15]{figuras/costes-mantenimiento}
                \only<3>{\caption{Imagen generada con IA.}}
                \includegraphics<4>[scale=0.35]{figuras/portabilidad}
            \end{figure}
        \end{column}
    \end{columns}
\end{frame}

\begin{frame}
    \frametitle{¿Qué se propone?}
    {\it Dagger} para gestionar CI/CD.
    \begin{columns}
        \hfill
        \begin{column}{0.55\textwidth}
            \begin{itemize}
                \item<1-> ¿Qué es Dagger?
                \item<2-> ¿Cómo de complejo es?
                    % curva de aprendizaje
                \item<3-> ¿Facilita el desarrollo?
                    % comparado con tecnologías tradicionales
                \item<4-> ¿Disminuye el coste de mantenimiento?
                    % del que se habló antes
                \item<5-> ¿Mejora la portabilidad?
            \end{itemize}
        \end{column}
        \begin{column}{0.45\textwidth}
            \begin{figure}
                \hspace*{-1.8cm}\includegraphics<1->[scale=1.2]{figuras/Dagger_logo}
            \end{figure}
        \end{column}
    \end{columns}
\end{frame}

\section{Fundamentos teóricos}
\subsection{Conceptos y herramientas}
\begin{frame}
    \frametitle{Conceptos básicos}
    \begin{itemize}
        \item CI/CD (\textit{Continuous Integration/Continuous Delivery}). \pause
        \item GitOps \& ArgoCD. \pause
        \item Kubernetes \& Helm.
    \end{itemize}
\end{frame}
\subsection{Dagger}

\begin{frame}
    \frametitle{Dagger}
    \begin{itemize}
        \item SDK de creación de \textit{pipelines} de CI/CD. \pause
        \item Múltiples lenguajes. \pause
        \item Portabilidad. \pause
        \item Módulos.
    \end{itemize}
\end{frame}

\section{Bibliografía}
\begin{frame}
    \frametitle{Bibliografía}
    \begin{thebibliography}{99}
        \scriptsize
        \bibitem{gitlab_architecture} GitLab Inc. ``GitLab Architecture Overview | GitLab Docs.'' Gitlab.com, 2025, {\it https://docs.gitlab.com/development/architecture/\#simplified-component-overview}. Accedido el 26 de agosto del 2025.
        \bibitem{gitlab_repo} GitLab Inc. ``GitLab.org / GitLab.'' GitLab, 8 Feb. 2020, {\it https://gitlab.com/gitlab-org/gitlab}.
        \bibitem{gitlab_ci} GitLab Inc. ``Doc/Development/Cicd/\_index.md · Master · GitLab.org / GitLab · GitLab.'' GitLab, 2025, {\it https://gitlab.com/gitlab-org/gitlab/-/blob/master/doc/development/cicd/\_index.md}. Accedido el 26 de agosto del 2025.
    \end{thebibliography}
\end{frame}

\end{document}
% Dagger functions (encadenamiento)
% Daggerverse

% \section{Sección 1}
% \subsection{Ejemplo de subsección}
% \begin{frame}
%     \frametitle{Título del frame 1}
%     Cada pantalla tiene su título.
% \end{frame}
%
% \subsection{Ejemplo de lista}
%
% \begin{frame}
%     \frametitle{Lista no numerada}
%     \begin{itemize}
%         \item una
%         \item dos
%         \item tres
%         \item cuatro
%     \end{itemize}
% \end{frame}
%
% \begin{frame}
%     \frametitle{Lista con pausa}
%     \begin{itemize}
%         \item número uno \pause
%         \item número dos \pause
%         \item número tres \pause
%         \item número cuatro
%     \end{itemize}
% \end{frame}
%
% \subsection{Otro ejemplo de lista}
% \begin{frame}
%     \frametitle{Lista numerada}
%     \begin{enumerate}
%         \item una
%         \item dos
%         \item tres
%         \item cuatro
%     \end{enumerate}
% \end{frame}
%
% \section{Sección 2}
% \subsection{Tablas}
%
% \begin{frame}
%     \frametitle{Tablas}
%     \begin{tabular}{|c|l|r|} \hline
%         \textbf{Centrado} & \textbf{Izquierda} & \textbf{Derecha} \\ \hline
%         AAAA              & 1000               & aaaa             \\ \hline
%         BB                & 20                 & bb               \\ \hline
%     \end{tabular}
% \end{frame}
%
% \begin{frame}
%     \frametitle{Tabla con pausa}
%     \begin{tabular}{c c c}
%         A & B & C \\ \pause
%         1 & 2 & 3 \\  \pause
%         A & B & C \\
%     \end{tabular}
% \end{frame}
%
% \section{Sección 3}
% \subsection{Bloques}
%
% \begin{frame}
%     \frametitle{Bloques}
%
%     \begin{block}{Bloque normal}
%         Texto del bloque normal
%     \end{block}
%
%     \begin{exampleblock}{Bloque de ejemplo}
%         Texto del bloque ejemplo
%     \end{exampleblock}
%
%     \begin{alertblock}{Bloque de alerta}
%         Texto del bloque alerta
%     \end{alertblock}
% \end{frame}
%
% \section{Sección 4}
% \subsection{Pantalla dividida}
%
% \begin{frame}
%     \frametitle{Pantalla dividida}
%     \begin{columns}
%         \begin{column}{5cm}
%             \begin{itemize}
%                 \item una lista
%                 \item de puntos
%                 \item mas una tabla
%             \end{itemize}
%         \end{column}
%         \begin{column}{5cm}
%             \begin{tabular}{|c|c|c|} \hline
%                 \textbf{Mes} & \textbf{Día} & \textbf{Hora} \\ \hline
%                 Enero        & 10           & 15:30         \\ \hline
%                 Febrero      & 20           & 20:00         \\ \hline
%             \end{tabular}
%         \end{column}
%     \end{columns}
% \end{frame}
%
% \subsection{Figuras}
% \begin{frame}
%     \frametitle{Incluir figuras}
%     \begin{figure}
%         \includegraphics[scale=0.3]{figuras/logo_usc.eps}
%         \caption{Logo de la USC}
%     \end{figure}
% \end{frame}
%
% \subsection{Listas con figuras y pausas}
%
% \begin{frame}
%     \frametitle{Listas con figuras y pausas}
%     \begin{columns}
%         \begin{column}{4cm}
%             \begin{itemize}
%                 \item<1-> Una
%                 \item<3-> Dos
%                 \item<5-> Tres
%             \end{itemize}
%             \vspace{3cm}
%         \end{column}
%         \begin{column}{4cm}
%             \begin{overprint}
%                 \includegraphics<2>[scale=0.05]{figuras/logo_usc.eps}
%                 \includegraphics<4>[scale=0.10]{figuras/logo_usc.eps}
%                 \includegraphics<6>[scale=0.15]{figuras/logo_usc.eps}
%             \end{overprint}
%         \end{column}
%     \end{columns}
% \end{frame}
%
% \subsection{Cuando se necesita más espacio}
% \begin{frame}[plain]
%     \frametitle{Pantalla plana con sólo una figura}
%     \begin{figure}
%         \includegraphics[scale=0.3]{figuras/figura01.eps}
%         \caption{Una figura grande}
%     \end{figure}
% \end{frame}
